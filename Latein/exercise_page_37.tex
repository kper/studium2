\documentclass[a4paper]{article}
\usepackage[T1]{fontenc}
\usepackage[ngerman]{babel}

\usepackage[margin=1in]{geometry}
\usepackage{lscape}
\pagenumbering{gobble}
\usepackage{amsmath}

\setlength{\parindent}{0em}

\newcommand{\ann}[2]{$\underbrace{\text{#1}}_{#2}$}
\newcommand{\ovv}[2]{$\overbrace{\text{#1}}^{#2}$}

\begin{document}

Lectio 5 \\

1. Postquam \ann{Quintus}{Nominativ Sing} \ann{Marcum}{Akk Sing} \ann{delicti magni}{Genetiv Sing} \ovv{accusavit}{3P Sing Perfekt}, \ann{Marcus}{Nominativ Sing} \ovv{fugere}{Präsens Indikativ} \ovv{temptavit}{3P Sing Präsens Aktiv}. \\

Marcus hat versucht zu fliehen, nachdem Quintus Marcus eines großen Delikts beschuldigte. \\

2. Cum \ann{negotium}{Akk Sing} cum \ann{donatione}{Ablativ Sing} \ovv{miscetur}{3P Präsens Passiv}, \ann{fiscus}{Nominativ Sing} saepe \ovv{eluditur}{3P Präsens Passiv} . \\

Die Steuer wird ausgetrickst, wenn Schenkung und Geschäft vertauscht wird. \\

3. \ann{Vir innocens}{Nominativ Sing} si \ovv{accusatur}{3P Singular Präsens Passiv} \ovv{puniturque}{3P Präsens Passiv Singular}, pro \ann{damno}{Ablativ} suo \ovv{resarciri}{Präsens Infinitiv} \ovv{debet}{3P Präsens Aktiv}. \\


Wenn ein unschuldiger Mann verurteilt und bestraft wird, dann muss ihm wiedergut gemacht werden für die Verurteilung. \\

4. Quamquam multi de \ann{urbium}{Genetiv} \ovv{originibus}{Dativ Pl} \ovv{scripserunt}{PPP 3P Pl}, de Romae origine fabulas tantum \ovv{habemus}{1P Pl Präsens Aktiv}. \\

Obwohl viele über Ursprünge großer Städte geschrieben haben, haben wir über Roms Ursprung nur Mythen. \\

Lectio 10

1. Illa \ann{in causa magna}{Ablativ Singular} \ann{reorum}{Genetiv Pl} pars \ann{a iudicibus}{Ablativ} damnata \ovv{est}{3P Präsens Singular}. \\

Hmmm \\

2. \ann{Fraudis accusatis}{Genetiv Sing} \ann{corpora}{Akk Pl} {delicti}{Genetiv Singular} hoc in foro \ovv{ostendebantur}{Imperfekt 3P Pl Aktiv}. \\

Es wurden dem beschuldigten Betrüger die Beweisgegenstände des Delikts entgegengesetzt.\\

3. Illis temporibus iudices rei publicae legibus non scriptis usi erant. \\

Hmmm. \\

4. \ann{Omnes iudices}{Nominativ} \ann{suae rei publicae legibus}{Dativ Pl} uti \ovv{obligantur}{3P Pl Passiv Präsens}. \\

Alle Richter müssen sich an die Gesetze der Republik verpflichten.





\end{document}

\documentclass[a4paper]{article}
\usepackage[T1]{fontenc}
\usepackage[ngerman]{babel}

\usepackage[margin=1in]{geometry}
\usepackage{lscape}
\pagenumbering{gobble}
\usepackage{amsmath}

\newcommand{\ann}[2]{$\underbrace{\text{#1}}_{#2}$}
\newcommand{\ovv}[2]{$\overbrace{\text{#1}}^{#2}$}

\begin{document}
1. \ann{Titus}{Nominativ} \ovv{vidit}{3P Singular Perfekt Aktiv} \ann{Cornelium}{Dativ}. Et \ann{Cornelius}{Nominativ} \ovv{vidit}{3P Singular Perfekt Aktiv} \ann{Titum}{Dativ}.\\

Titus hat Cornelius gesehen. Und Cornelius hat Titus gesehen. \\

2. \ann{Servus}{Nominativ Sing} \ann{domino}{Dativ} nuntium \ovv{apportavit.}{3P Singular Perfekt Aktiv}  \\

Der Sklave hat die Nachricht zum Herren getragen. \\

3. \ann{Dominus}{Nominativ Sing} \ann{bonum servum}{Dativ} \ovv{liberaverat}{3P Singular Imperfekt Aktiv}. \\

Der Herr befreite den guten Sklaven. \\

4. \ann{Reos}{AkkusativSing } \ovv{damnavistis}{2P Plural Perfekt Aktiv} \\

Ihr habt den Angeklagten verurteilt. \\

5. \ann{Servum}{Akkusativ Sing} \ann{filii}{Genetiv Singular} \ovv{vidisti}{2P Singular Perfekt Aktiv}. \\

Du hast den Sklaven des Sohnes gesehen. \\

6. \ann{Servum}{Akkusativ Sing} \ann{filii}{Genetiv Singular} \ovv{viderunt}{3P Plural PPP}. \\

Sie hatten den Sklaven des Sohnes gesehen. \\

7. \ann{Servorum}{Genetiv Sing} bonorum \ann{filios}{Akkusativ Plural} \ovv{vidisti.}{2P Singular Perfekt Aktiv} \\

Du hast die Kinder des guten Sklaven gesehen. \\

8. \ann{Amici}{Genetiv Sing} \ann{causam}{1P Singular} \ovv{egeram}{1P Singular PPP Aktiv}. \\

Ich hatte die Klage der Freunde betrieben. \\

9. \ann{Amici}{Genetiv Sing} \ann{causam}{1P Singular} \ovv{egerant}{3P Plural PPP Aktiv}. \\

Sie hatten die Klage der Freunde betrieben. \\

Übungstext: \\

1. \ann{Gaius}{Nominativ} \ann{Tito}{Dativ} magnum \ann{iniuriam}{Akkusativ Sing} \ovv{fecerat}{3P Singular Perfekt Aktiv}. \\

Gaius hat Tito ein großes Unrecht getan. \\

2. \ann{Titus}{Nominativ} \ann{adversarium}{Akkusativ Sing} \ann{delicti}{Genetiv Singular} \ovv{accusavit}{3P Singular Perfekt Aktiv}. \\

Titus hat den Gegner des Verbrechens angeklagt. \\

3. \ann{Iudex}{Nominativ} \ann{rei}{Akkusativ Sing} \ann{causam}{Akkusativ Sing} \ovv{audivit}{3P Singular Perfekt Aktiv} et \ann{reum}{Akkusativ Sing} \ovv{damnavit}{3P Singular Perfekt Aktiv}. \\

Der Richter hat die Klage des Angeklagten gehört und er verurteilte ihn.\\

4. \ann{Iudex}{Nominativ} \ann{reo}{Dativ} \ovv{dixit}{3P Singular Perfekt Aktiv}: "Contra facta non valent argumenta. Ergo \ovv{puniri}{passiv Infinitiv} \ovv{debes}{2P Singular Präsens Aktiv}."

Der Richter hat zum Angeklagten gesprochen: "Keine validen Argumenten gegen die Fakten(?). Daher schuldest du durch Bestrafung (?)" \\

5. \ann{Iudex}{Nominativ} \ovv{reum}{Akkusativ} non \ovv{absolvit}{3P Singular Perfekt Aktiv}. \\ 

Der Richter hat den Angeklagten nicht freigesprochen. \\

6. \ovv{Reus}{Nominativ} \ann{causam}{Akkusativ} \ovv{dixerat}{3P Singular PPP Aktiv}, sed \ann{causam}{Akkusativ} non \ovv{tenuerat}{3P Singular PPP Aktiv}. \\

Der Angeklagte hatte zur Klage gesprochen, aber es hatte nicht gehalten.






\end{document}

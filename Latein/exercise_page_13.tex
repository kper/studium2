\documentclass[a4paper]{article}
\usepackage[T1]{fontenc}
\usepackage[ngerman]{babel}

\usepackage[margin=1in]{geometry}
\usepackage{lscape}
\pagenumbering{gobble}
\usepackage{amsmath}

\setlength{\parindent}{0em}

\newcommand{\ann}[2]{$\underbrace{\text{#1}}_{#2}$}
\newcommand{\ovv}[2]{$\overbrace{\text{#1}}^{#2}$}

\begin{document}

1. \ovv{Silent}{Präsens 3P Sing Aktiv} \ann{leges}{Nominativ Pl} inter \ann{arma}{Genetiv}. \\

Die Gesetze schweigen in Zeiten des Krieges. \\

2. \ann{Leges}{Akk Pl} inter arma \ann{silere}{Infinitiv} \ovv{scitis}{2P Sing Perfekt Aktiv}. \\

Du hast gewusst, dass die Gesetze schweigen in Zeiten des Krieges. \\

3. \ann{Legibus}{Akk Pl} \ovv{obligamur}{Präsens 1P Pl Passiv} \\

Wir sind den Gesetzen verpflichtet. \\

4. \ann{Nos legibus}{Akk Pl} \ovv{obligari}{Präsens Infinitiv} \ovv{scimus}{Präsens 1P Pl Aktiv}. \\

Wir wissen, dass Gesetze uns verpflichten. \\

5. \ann{Imperatori}{Dativ Pl} \ann{plebs}{Nominativ} \ann{honores magnos}{Akk Pl} \ovv{tribuit.}{3P Perfekt aktiv} \\

Die Bauern haben dem Herrscher eine große Ehre erwiesen. \\

6. \ann{Originem}{Akk Sing} \ann{urbium}{Genetiv Pl} saepe \ovv{ignoramus.}{Präsens 1P Pl Aktiv} \\

Wir ignorieren oft die Herkunft der Städte. \\

7. ehhh

8. \ann{Cives}{Nominativ Pl} \ann{iniuria}{Ablativ Sing} \ovv{vexabantur}{Imperfekt 3P Pl Aktiv}. \\

Die Bevölkerung wurde durch ein Unrecht heimgesucht. \\

9. Ab \ann{imperatoribus}{Ablativ Pl} \ann{Romanis}{Nominativ Pl} \ann{multa bella}{Akku Pl} \ovv{gerebantur.}{Imperfekt 3P Pl Aktiv} \

Die Römer mussten mehrere Kriege aufgrund ihrer Herrscher ertragen. \\

10. Minima non \ovv{curat}{Präsens 3P Sing Aktiv} \ann{praetor}{Nominativ Sing}. \\

Der Vorsteher

\end{document}

\documentclass[a4paper]{article}
\usepackage[T1]{fontenc}
\usepackage[ngerman]{babel}

\usepackage[margin=1in]{geometry}
\usepackage{lscape}
\pagenumbering{gobble}
\usepackage{amsmath}

\setlength{\parindent}{0em}

\newcommand{\ann}[2]{$\underbrace{\text{#1}}_{#2}$}
\newcommand{\ovv}[2]{$\overbrace{\text{#1}}^{#2}$}

\begin{document}

1. Ich sehe, dass die Freunde kommen. \\
2. Ich sehe, dass die Freunde gekommen sind. \\
3. Wir weiß, dass das Verbrechen bestraft wird. \\
4. Er weiß, was das Verbrechen des Angeklagten macht. \\
5. Wir sehen, dass Marcus ein neues Haus baut. \\
6. Ich hörte, dass Claudius beschuldigt ist. \\
7. Ich habe gute Bücher gelesen. Ich glaube, dass die großen Bücher gelesen werden. \\
8. Der Richter vermutet, dass der Angeklagte sich schuldig gemacht hatte.

\end{document}

\documentclass[a4paper]{article}
\usepackage{geometry}
\geometry{margin=1in}
\usepackage[ngerman]{babel}
\usepackage{lmodern}
\usepackage{amsfonts}
\usepackage{amssymb}
\usepackage{amsmath}
\usepackage{csquotes}
\usepackage[x11names]{xcolor}
\usepackage{framed}
\usepackage{quoting}
\usepackage{listings}
\usepackage[utf8]{inputenc}
\usepackage{bussproofs}
\setlength{\parindent}{0cm}

\colorlet{shadecolor}{LavenderBlush2}
\usepackage{lipsum}
\newenvironment{shadedquotation}
 {\begin{shaded*}
  \quoting[leftmargin=0pt, vskip=0pt]
 }
 {\endquoting
 \end{shaded*}
}

\newcommand\myt[1]{[$\boldsymbol{T_#1}$]}

\begin{document}

Keine Garantie auf Korrektheit \\

\begin{shadedquotation}
Der reiche Crassus schenkt und übergibt der schönen Messalina auf deren Bitte eine Halskette. \\
Messalina legt sich die Kette sofort um. \\

a. Diskutieren Sie die Frage des Besitzerwerbes der Messalina unter der Annahme, dass Crassus ein \textit{pupillus} ist.
\end{shadedquotation}


Crassus ist ein Pupillus. Damit ist es nicht möglich ohne seinen Tutor Animus aufgeben, auch wenn er Corpus verliert. \\

Dadurch kann Messalina nur einen Fremdbesitzwillen haben und das Anlegen der Kette besitzt sie nun auch Corpus und für Crassus wird der Corpus mittelbar. Damit ist sie Fremdbesitzerin.

\begin{shadedquotation}
Der reiche Crassus schenkt und übergibt der schönen Messalina auf deren Bitte eine Halskette. \\
Messalina legt sich die Kette sofort um. \\

b. Diskutieren Sie die Frage des Besitzerwerbes der Messalina unter der Annahme, dass Messalina ein \textit{pupilla} ist.
\end{shadedquotation}

Crassus gibt seinen Animus auf und durch das Anlegen der Kette geht Corpus an Messalina über. Als Pupilla kann sie nur Geschenke annehmen, die zu ihrem Vorteil sind. \\

Durch die explizite Bitte fasst sie Animus. Da ihr die Kette angelgt wurde, hat sie Corpus und damit Besitz erworben.



\end{document}
